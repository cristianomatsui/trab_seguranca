\section{Geração de Sequências Aleatórias}
Em geral, sistemas computacionais são determinísticos, ou seja, cada evento é definido pela sequência de eventos precedentes. Essa é basicamente a definição contrária à definição de aleatoriedade, criando assim um problema na geração de números aleatórios por computadores. Porem, existem duas maneiras principais de se resolver esse problema.
Uma forma é a utilização de sequências pseudoaleatórias. Essas sequências se baseiam em uma série de operações determinísticas que , a partir de um conjunto de parâmetros, são capazes de gerar sequências que se pareçam suficientemente aleatórias para alguma finalidade específica. Esse método tem algumas vantagens, como uma grande eficiência, gerando grandes sequencias em um curto período de tempo, e reprodutibilidade, que permite a repetição de testes, uma vez que o mesmo conjunto de parâmetros inicias gera a mesma sequência
Porem, em muitos casos, como para criptografia, a reprodutibilidade dos dados não é desejada, pois os valores devem ser realmente imprevisíveis. Para tal, existem os chamados geradores de números verdadeiramente aleatórios, que devido à natureza determinística de sistemas computacionais, necessitam de obtenção de dados externos ao computador. 
É comum geradores pseudoaleatórios se utilizarem de algum dado externo como parâmetro para a geração de suas sequências, como o tempo do sistema, de forma à variar seus resultados. Já geradores de sequências numéricas verdadeiramente aleatórios, dependem de uma fonte de dados aleatórios para a geração da sequência, dados estes que devem apresentar alguma forma de aleatoriedade intrínseca. Assim, geradores de números verdadeiramente aleatórios costumam se utilizar da medição de fenômenos físicos, como sons atmosféricos ou até mesmo decaimento radioativo.
