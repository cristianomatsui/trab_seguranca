\section{Introdução}
Um número aleatório é um elemento de uma sequência numérica que apresenta aleatoriedade estatística, ou seja, uma sequencia que não apresente nenhuma regularidade ou padrão, de forma que o valor de um elemento qualquer não possa ser previsto a partir dos valores prévios na sequência.
Na área computacional, a utilização de números aleatórios é de grande utilidade em diversas situações, como em simulações de situações reais, que contenha fatores externos aos controlados, sistemas de jogos de apostas, que devem ter resultados imprevisíveis à qualquer pessoa, e no contexto de segurança, na geração de chaves criptográficas.
Porem, sem uma fonte de dados aleatórios, um computador não é capaz de gerar sequências verdadeiramente aleatórias, e acaba se baseando em sistemas de gerações pseudoaleatórias. Este projeto objetiva a criação de um sistema capaz de gerar sequências numéricas verdadeiramente aleatórias, a partir da interferência (ruído) eletromagnética captada por um microcontrolador.