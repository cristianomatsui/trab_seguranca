\documentclass[
% -- opções da classe memoir --
12pt,				% tamanho da fonte
openright,			% capítulos começam em pág ímpar (insere página vazia caso preciso)
twoside,			% para impressão em frente e verso. Oposto a oneside
a4paper,			% tamanho do papel. 
% -- opções da classe abntex2 --
%chapter=TITLE,		% títulos de capítulos convertidos em letras maiúsculas
%section=TITLE,		% títulos de seções convertidos em letras maiúsculas
%subsection=TITLE,	% títulos de subseções convertidos em letras maiúsculas
sumario=tradicional, % configura estilo do sumário (comente estilo ABNTEX)
%subsubsection=TITLE,% títulos de subsubseções convertidos em letras maiúsculas
% -- opções da classe hyperref
hidelinks,          % oculta caixas e links coloridos
% -- opção de numerar continuamente as equações
%continuouseq,      % descomente para numeração contínua
% -- opção para citações e referências padrão IEEE	
%IEEE,				% citações e referências IEEE
% -- opções para citação utilizando classe abnt2cite
%alf,				% citação alfanumérica
num,				% citação numérica 
%overcite,			% citação no modo sobrescrito (se usar notas de rodapé, evite)
bibjustif,			% alinhamento justificado para as referências
brackets,			% citações com delimitador []
% -- opções do pacote babel --
english,			% idioma adicional para hifenização
brazil				% o último idioma é o principal do documento
]{article}       % utfprcopel.cls based on abnt2.cls


\usepackage{sbc-template}

\usepackage{graphicx,url}

\usepackage[portuguese, ruled, linesnumbered]{algorithm2e}
\usepackage[brazil]{babel}
\usepackage[utf8]{inputenc} 
\usepackage{indentfirst}
     
\sloppy

\title{Obtenção de sequências aleatórias a partir de interferência eletromagnetica}

\author{Cristiano M. Matsui\inst{1}, Kallil M. Caparroz\inst{2}, Rodrigo C. Anater\inst{3} }


\address{Universidade Tecnológica Federal do Paraná - Câmpus Pato Branco
  \email{amargo@gmail.com, kallil@alunos.utfpr.edu.br,
  rodrigoanater@alunos.utfpr.edu.br}
}

\begin{document} 

  \selectlanguage{brazil}
  \maketitle

\begin{abstract} 	
 	The definition of what is meant by a random number or event is the subject of discussion by professionals in many fields of study, including computing professionals. However, most part of the computers used in the present days generate only pseudorandom numbers. This article details the development of a random number generator using the electrical noise obtained from the analog port of a microcontroller
\end{abstract}
     
\begin{resumo} 
 	A definição do que se entende por um número ou evento aleatório é alvo de debates por parte de profissionais de diversas áreas de estudo, inclusive profissionais de computação. Entretanto, a maioria dos computadores presentes na atualidade geram apenas números pseudo-aleatórios. O presente artigo detalha o desenvolvimento de um gerador de números aleatórios utilizando o ruído elétrico obtido da porta analógica de um microconrolador.  
\end{resumo}


\section{Introdução}



\bibliographystyle{sbc}
\bibliography{sbc-template}
\end{document}